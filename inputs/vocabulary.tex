\newcommand\use[1]{\input{inputs/vocabulary_terms/#1.tex}}
\newcommand\row[3]{{#1} & {#2} & {\use{#3}}\\ \hline}
\newcommand\rowNoLine[3]{{#1} & {#2} & {\use{#3}}\\}
\newcommand\rowit[4]{{#1 \textit{#2}} & {#3} & {\use{#4}}\\ \hline}
\newcommand\rowitNoLine[4]{{#1 \textit{#2}} & {#3} & {\use{#4}}\\}
\section*{\MakeUppercase{Словарь терминов СКС}}

\begin{longtable}[c]{|p{3cm}|p{3cm}|p{4.5cm}|}
    \hline
    {\bf Английские} & {\bf Русские термины} & {\bf Толкование}\\
    {\bf термины} & {} & {}\\
    \hline

    \row{Access floor}{Фальшпол}{access_floor}
    \row{Active equipment}{Активное оборудование}{active_equipment}
    \row{Adapter}{Адаптер}{adapter_1}
    \row{}{Соединитель}{adapter_2}
    \row{Administration}{Администрирование}{administration}
    \row{Aerial cable}{Воздушный кабель}{aerial_cable}
    \row{Air blown fiber}{Пневматическая прокладка волокна}{air_blown_fiber}
    \row{Alien crosstalk}{Межкабельные наводки}{alien_crosstalk}
    \row{Alternate entrance}{Резервный вход}{alternate_entrance}
    \row{Antenna entrance}{Антенный вход}{antenna_entrance}
    \row{Application}{Приложение}{application}
    \row{Array connector}{Многоволоконный разъем}{array_connector}
    \row{Attenuation}{Затухание}{attenuation}
    \row{Attenuation crosstalk ratio (ACR)}{Отношение затухания к двунаправленным (перекрестным) наводкам}{acr}


    \row{Backbone}{Магистраль}{backbone}
    \rowit{Backbone bonding conductor}{синоним -- telecommunications bonding backbone}{Телекоммуника-ционная магистраль заземления}{backbone_bonding}
    \row{Backbone cable}{Магистральный кабель}{backbone_cable}
    \row{Backbone pathway (raceway)}{Магистральный кабелепровод}{backbone_pathway}
    \row{Backward compatibility}{Совместимость категорий}{backward}
    \row{Balanced cable}{Симметричный кабель}{balanced_cable}
    \row{Balanced transmission}{Сбалансированная передача}{balanced_transmission}
    \row{Balun (balanced-unbalanced)}{Волновой адаптер}{balun}
    \row{Bandwidth}{Полоса пропускания}{bandwidth}
    \row{Basic link}{Базовая линия}{basic}
    \row{Bonding}{Соединение}{bonding}
    \row{Bonding conductor for telecommunications}{Соединяющий проводник телекоммуникаций}{bondingcond}
    \row{Building backbone cable}{Кабель магистрали здания}{building_bbc}
    \rowit{Building distributor}{(международный стандарт)}{Распределительный пункт здания}{building_dist}
    \row{Building entrance facility}{Пункт ввода в здание}{building_entrance}
    \row{Bundle}{Связка}{bundle}
    \row{Bus architecture}{Шинная архитектура}{bus}


    \row{Cabinet}{Шкаф}{cabinet}
    \row{Cable}{Кабель}{cable}
    \row{Cable element}{Кабельный элемент}{cable_e}
    \row{Cable unit}{Кабельная единица, сборка}{cable_u}
    \row{Cabling}{Кабельная система}{cabling}
    \row{Cabling system certification}{Сертификация СКС}{csc}
    \row{Campus}{Комплекс зданий}{campus}
    \row{Campus backbone}{Магистраль комплекса}{campus_bb}
    \row{Campus distributor}{Распределительный пункт комплекса}{campus_d}
    \rowit{Category}{(американский стандарт)}{Категория}{category}
    \row{Ceiling distribution system}{Система потолочной прокладки}{cds}
    \row{Channel}{Канал}{channel}
    \rowit{Class}{(международный стандарт)}{Класс}{class}
    \row{Collapsed backbone}{Абонентская магистраль}{collapsed}
    \row{Commercial building}{Коммерческое здание}{commercial}
    \row{Component compability}{Совместимость элементов}{component}
    \row{Conduit}{Трубопровод}{conduit}
    \row{Connecting hardware}{Разъемные элементы}{connecting_h}
    \rowNoLine{Connector}{Разъем}{connector_1}
    \row{}{Коннектор}{connector_2}
    \row{Consolidation pont}{Точка консолидации}{consolidation}
    \row{Cord, telecommunications}{Соединительный кабель}{cord}
    \rowit{CP cable}{(проект международного стандарта)}{Переходный кабель}{cp_cable}
    \row{Cross connect}{Коммутационный пункт}{cross_connect}
    \row{Cross connection}{Коммутируемое подключение}{cross_connection}
    \row{Cross connect panels}{Коммутационные панели}{cross_con_p}
    \row{Crossover}{Пересечение}{crossover}
    \row{Crossover adapter}{Перекрестный адаптер}{crossover_a}
    \row{Crossover cable}{Перекрестный кабель}{crossover_c}
    \rowit{Cut down}{синноним - punch down}{Врезка}{cut_down}


    \row{Delay}{Задержка}{delay}
    \row{Demo rack}{Демонстрационная стойка}{demo}
    \row{Direct buried}{Траншейный}{direct}
    \row{Dispersion}{Дисперсия}{dispersion}
    \row{Distribution frame, DF}{Кросс}{df}
    \row{Distributor}{Распределительный пункт}{distributor}
    \row{Dual fibre cable}{Двухволоконный оптический кабель}{dual}
    \rowNoLine{Duct}{Кабельный канал}{duct_1}
    \row{}{Воздухопровод}{duct_2}


    \row{Effective ground}{Эффективное заземление}{effective}
    \row{Electrical closet}{Электрощитовая}{electrical_closet}
    \row{Electromagnetic compatibility (EMC)}{Электромагнитная совместимость (ЭМС)}{emc}
    \row{Equal Level Far End Crosstalk (ELFEXT)}{Отношение затухания к однонаправленным наводкам}{elfext}
    \row{Entrance facility}{Ввод в здание}{entrance_f}
    \row{Entrance point}{Точка ввода}{entrance_p}
    \row{Entrance room or space for telecommunicaton}{Помещение ввода телекоммуникаций}{entrance_r}
    \row{Equipment cable}{Сетевой кабель}{equip_c}
    \row{Equipment room}{Комната для оборудования, аппаратная}{equip_r}
    \row{Ethernet}{{Ethernet, ""Эзернет""}}{ethernet}
    \row{Euromod}{Евромод}{euromod}
    \row{Exotermic weld}{Экзотермическая сварка}{exotermic}


    \row{Female connector}{Гнездовой разъем}{female}
    \row{Far End Crosstalk}{Однонаправленные наводки}{fext}
    \row{Fiber to the desk}{Абонентский волоконно-оптический канал}{fttd}
    \row{Fiber management system (FMS)}{Системы организации волокон (СОВ)}{fms}
    \row{Firestop}{Противопожарная заглушка}{firestop}
    \row{Floor distributor}{Распределительный пункт этажа}{floor}
    \row{Flush mount inserts}{Вставки заподлицо}{flush}
    \row{Foiled Twisted Pair (FTP)}{Фольгированная витая пара (ФВП)}{ftp}
    \row{Foiled and Braided Twisted Pair (FBTP)}{Фольгированная с оплеткой витая пара (ФОВП)}{fbtp}
    \row{Furniture cluster}{Блок рабочих мест}{furniture}


    \row{Generic cabling}{Структурированная кабельная система (СКС)}{generic_c}
    \row{Generic interface}{Структурированный интефейс}{generic_i}
    \row{Grade index fiber}{Градиентное волокно}{grade}
    \row{Grounding electrode}{Заземляющий электрод}{ground_e}
    \row{Grounding electrode conductor}{Проводник заземляющего электрода}{ground_e_c}


    \row{Horizontal cabe}{Горизонтальный кабель}{hori_cabe}
    \row{Horizontal cabling}{Горизонтальная подсистема}{hori_cabling}
    \rowit{Horizontal cross connect}{(ам. стандарт)}{Пункт коммутации горизонтальной подсистемы}{hori_x}
    \row{Hybrid cable}{Гибридный кабель}{hybrid}


    \row{Identifier}{Обозначение, идентификатор}{id}
    \row{Impedance}{Волновое сопротивление, импеданс}{im}
    \row{In-conduit}{В канализации}{in}
    \row{Individual work area}{Персональная рабочая область}{ind}
    \row{Insulation displacement contact(IDC)}{Врезной контакт, контакт сквозь изоляцию (КСИ)}{ins}
    \row{Integrated Services Digital Network (ISDN)}{Цифровая сеть с интеграцией служб (ЦСИС)}{isdn}
    \row{Interbuilding cable}{Кабель между зданиями, внешний кабель}{inter_c}
    \row{Interbuilding cable entrance}{Ввод кабеля в здание}{inter_ce}
    \rowit{Interbuilding cabling}{межд. стандарт}{Кабельная система между зданиями}{inter_cing}
    \rowit{Interconnect}{международный стандарт}{Подключение}{inter_con}
    \row{Interconnect panel}{Соединительная панель}{inter_con_p}
    \rowitNoLine{Interconnection}{американский стандарт}{Соединение}{interconnection}
    \row{}{Подключение}{interconnection2}
    \row{Interface}{Интерфейс}{interface}
    \rowit{Intermediate cross connect}{американский стандарт}{Промежуточный коммутационный пункт}{icc}


    \row{Jack}{Гнездо (гнездовой разъем)}{jack}
    \row{Jumper}{Перемычка}{jumper}


    \row{Keying}{(Ключ-) совмещение}{keying}


    \row{Link}{Линия}{link}
    \row{Linkage}{Ссылка}{linkage}
    \row{Local Area Network (LAN)}{Локальная сеть (ЛС)}{lan}
    \row{Low Smoke Zero Halogen (LSOH)}{Кабель с малодымной безгалогеновой оболочкой (МДБГ)}{lsoh}


    \rowit{Main cross connect}{(американский стандарт)}{Главный коммутационный пункт}{mcc}
    \row{Main distribution frame, MDF}{Главный кросс}{mdf}
    \row{Main grounding busbar}{Центральный терминал заземления}{mgb}
    \row{Media telecommunications*}{Среда передачи}{mt}
    \row{Mode}{Мода}{mode}
    \row{Modular jack}{Модульное гнездо}{mj}
    \row{Modular plug}{Модульный штекер}{mp}
    \row{Multimode optical fibre}{Многомодовое оптическое волокно}{mmof}
    \row{Multi-user telecommunications outlet assembly}{Многопользова-тельский набор розеток}{mutoa}

    
    \row{NEXT (Near End Crosstalk)}{Двунаправленные наводки}{next}
    \row{Nominal velocity of propagation (NVP)}{Номинальная скорость распространения (НСР)}{nvp}
    \row{Numerical aperture}{Числовая апаратура}{na}
    
    
    \row{Octopus}{Разветвитель}{octo}
    \row{Optical fibre}{Оптическое волокно}{of}
    \row{Optical fibre duplex adapter}{Волоконно-оптический дуплексный соединитель}{ofda}
    \row{Optical interconnect panel}{Оптическая соединительная панель}{oip}


    \row{Pair}{Пара}{pair}
    \row{Passive components}{Пассивные элементы}{passive}
    \rowit{Patch cord}{(американский стандарт)}{Коммутационный кабель}{pc_1}
    \rowit{Patch cord}{(международный стандарт)}{Коммутационный кабель}{pc_2}
    \row{Patch panel}{Коммутационная панель}{pp}
    \row{Plenum}{Воздухопровод}{plenum}
    \row{Plenum cable}{Кабель для воздуховода}{plenum_c}
    \row{Plug}{Штекер (штыревой разъем)}{plug}
    \row{Public network interface}{Интерфейс сети общего пользования}{pni}
    \row{Powersum (PS) ACR}{Отношение затухания к сумманым двунаправленным наводкам}{ps_asr}
    \row{Powersum (PS) ELFEXT}{Отношение затухания к суммарным однонаправленным наводкам}{ps_elfext}
    \row{Powersum (PS) FEXT}{Суммарные однонаправленные наводки}{ps_fext}
    \row{Powersum (PS) NEXT}{Суммарные двунаправленные наводки}{ps_next}
    \row{Premises distribution system (PDS)}{Распределительная система помещений (РСП)}{pds}
    \row{Protocol}{Протокол}{protocol}
    \row{Public network interface}{Интерфейс сети общего пользования}{pni_2}
    \row{Pull through cable}{Неразъемный кабель}{ptc}
    \rowit{Punch down,}{синоним -- cut down}{Забивка, врезка}{pd}


    \row{Quad}{Витая четверка}{quad}
    \row{Quad cable}{Четырехпроводный кабель}{quad_c}
    \row{Quad fibre cable}{Четырехволоконный кабель}{quad_fc}


    \row{Raceway}{Кабелепровод}{raceway}
    \row{Record}{Запись}{record}
    \row{Return loss}{Возвратные потери}{return}


    \row{Screened cable}{Экранированный кабель}{sc}
    \row{Screened twisted pair (ScTP)}{Экранированная витая пара (ЭВП)}{sctp}
    \row{Shielded twisted pair (STP)}{Защищенная витая пара (ЗВП)}{stp}
    \row{Shotgun cable}{Сдвоенный кабель}{sh_c}
    \row{Single mode fiber}{Одномодовое волокно}{smf}
    \row{Skew}{Фазовый сдвиг}{skew}
    \row{Sleeve}{Рукав, гильза}{sleeve}
    \row{Slot}{Паз}{slot}
    \row{Splice}{Сплайс}{splice}
    \row{Splice box}{Сплайсовая коробка}{splice_b}
    \row{Spring shut modular jack}{Модульное гнездо с подпружиненной шторкой}{ssmj}
    \row{SC adapter (SC -- subscriber connector)}{Соединитель ЭсСи}{sca}
    \row{ST adapter (ST -- strait tip)}{Соединитель ЭсТи}{sta}
    \row{Star topology}{Топология звезда}{st}
    \row{Step index fiber}{Ступенчатое волокно}{sif}
    \row{Straight cable}{Кабель прямого подключения, прямой кабель}{straight}
    \row{Straight tip connector}{Разъем типа ST (Эс Ти)}{straight_tc}
    \row{Structured cabling system}{Структурированная кабельная система (СКС)}{scs}
    \row{Support strand}{Опора}{ss}
    \row{Suspended ceiling}{Подвесной потолок}{suspended_c}
    \row{System Guaranty, Guaranty}{Системная гарантия}{sgg}


    \row{Telecommunications}{Телекоммуни-кации}{tc}
    \rowit{Telecommunications bonding backbone}{синоним -- backbone bonding conductor}{Телекоммуника-ционная магистраль заземления}{tcbb}
    \row{Telecommunications closet}{Телекоммуника-ционное помещение}{tcc}
    \row{Telecommunications infrastructure}{Телекоммуника-ционная инфраструктура}{tci}
    \row{Telecommunications grounding busbar}{Телекоммуника-ционная шина заземления}{tcgb}
    \row{Telecommunications outlet}{Телекоммуника-ционный разъем}{tco}
    \row{Telecommunications space}{Телекоммуника-ционное пространство}{tcs}
    \row{Terminal}{Терминал}{terminal}
    \row{}{Преобразователь}{terminal_1}
    \row{}{Разъем}{terminal_2}
    \row{Termination}{Оснащение кабельными разъемами, монтаж разъемов}{termination}
    \row{Transition point}{Точка перехода}{tp}
    \row{Transmission media}{Среда передачи}{tm}
    \row{Trough}{Лоток}{trough}
    \row{Twinaxial cable}{Двухосевой кабель}{twc}
    \row{Twisted pair}{Витая пара}{twp}


    \row{Unshielded twisted pair (UTP)}{Незащищенная витая пара (НЗВП)}{utp}
    \row{Usable floor space}{Используемая площадь}{ufs}
    \row{User code}{Код пользователя}{uc}
    \row{User's outlet}{Абонентская розетка}{uo}
    \row{Utility column}{Стойка}{ucol}


    \rowit{Vertical cabling}{(американский стандарт)}{Вертикальная кабельная система}{vertical_c}
    \row{Warranty}{Производственная гарантия}{warranty}
    \row{Wave Division Multiplexing}{Волновое мультиплексирование}{wave}
    \row{Wire}{Провод}{wire}
    \row{Work area}{Рабочая область}{work}
    \row{}{Рабочее место (смысловое знач.)}{work_2}
    \row{Work area cable}{Абонентский кабель}{work_ac}
\end{longtable}
