\newcommand\use[1]{\input{inputs/vocabulary_terms/#1.tex}}
\newcommand\row[3]{{#1} & {#2} & {\use{#3}}\\ \hline}
\newcommand\rowNoLine[3]{{#1} & {#2} & {\use{#3}}\\}
\newcommand\rowit[4]{{#1 \textit{#2}} & {#3} & {\use{#4}}\\ \hline}
\section*{\MakeUppercase{Словарь терминов СКС}}

\begin{longtable}[c]{|p{3cm}|p{3cm}|p{4cm}|}
    \hline
    {\bf Английские} & {\bf Русские термины} & {\bf Толкование}\\
    {\bf термины} & {} & {}\\
    \hline

    \row{Access floor}{Фальшпол}{access_floor}
    \row{Active equipment}{Активное оборудование}{active_equipment}
    \row{Adapter}{Адаптер}{adapter_1}
    \row{}{Соединитель}{adapter_2}
    \row{Administration}{Администрирование}{administration}
    \row{Aerial cable}{Воздушный кабель}{aerial_cable}
    \row{Air blown fiber}{Пневматическая прокладка волокна}{air_blown_fiber}
    \row{Alien crosstalk}{Межкабельные наводки}{alien_crosstalk}
    \row{Alternate entrance}{Резервный вход}{alternate_entrance}
    \row{Antenna entrance}{Антенный вход}{antenna_entrance}
    \row{Application}{Приложение}{application}
    \row{Array connector}{Многоволоконный разъем}{array_connector}
    \row{Attenuation}{Затухание}{attenuation}
    \row{Attenuation crosstalk ratio (ACR)}{Отношение затухания к двунаправленным (перекрестным) наводкам}{acr}


    \row{Backbone}{Магистраль}{backbone}
    \rowit{Backbone bonding conductor}{синоним -- telecommunications bonding backbone}{Телекоммуника-ционная магистраль заземления}{backbone_bonding}
    \row{Backbone cable}{Магистральный кабель}{backbone_cable}
    \row{Backbone pathway (raceway)}{Магистральный кабелепровод}{backbone_pathway}
    \row{Backward compatibility}{Совместимость категорий}{backward}
    \row{Balanced cable}{Симметричный кабель}{balanced_cable}
    \row{Balanced transmission}{Сбалансированная передача}{balanced_transmission}
    \row{Balun (balanced-unbalanced)}{Волновой адаптер}{balun}
    \row{Bandwidth}{Полоса пропускания}{bandwidth}
    \row{Basic link}{Базовая линия}{basic}
    \row{Bonding}{Соединение}{bonding}
    \row{Bonding conductor for telecommunications}{Соединяющий проводник телекоммуникаций}{bondingcond}
    \row{Building backbone cable}{Кабель магистрали здания}{building_bbc}
    \rowit{Building distributor}{(международный стандарт)}{Распределительный пункт здания}{building_dist}
    \row{Building entrance facility}{Пункт ввода в здание}{building_entrance}
    \row{Bundle}{Связка}{bundle}
    \row{Bus architecture}{Шинная архитектура}{bus}


    \row{Cabinet}{Шкаф}{cabinet}
    \row{Cable}{Кабель}{cable}
    \row{Cable element}{Кабельный элемент}{cable_e}
    \row{Cable unit}{Кабельная единица, сборка}{cable_u}
    \row{Cabling}{Кабельная система}{cabling}
    \row{Cabling system certification}{Сертификация СКС}{csc}
    \row{Campus}{Комплекс зданий}{campus}
    \row{Campus backbone}{Магистраль комплекса}{campus_bb}
    \row{Campus distributor}{Распределительный пункт комплекса}{campus_d}
    \rowit{Category}{(американский стандарт)}{Категория}{category}
    \row{Ceiling distribution system}{Система потолочной прокладки}{cds}
    \row{Channel}{Канал}{channel}
    \rowit{Class}{(международный стандарт)}{Класс}{class}
    \row{Collapsed backbone}{Абонентская магистраль}{collapsed}
    \row{Commercial building}{Коммерческое здание}{commercial}
    \row{Component compability}{Совместимость элементов}{component}
    \row{Conduit}{Трубопровод}{conduit}
    \row{Connecting hardware}{Разъемные элементы}{connecting_h}
    \rowNoLine{Connector}{Разъем}{connector_1}
    \row{}{Коннектор}{connector_2}
    \row{Consolidation pont}{Точка консолидации}{consolidation}
    \row{Cord, telecommunications}{Соединительный кабель}{cord}
    \rowit{CP cable}{(проект международного стандарта)}{Переходный кабель}{cp_cable}
    \row{Cross connect}{Коммутационный пункт}{cross_connect}
    \row{Cross connection}{Коммутируемое подключение}{cross_connection}
    \row{Cross connect panels}{Коммутационные панели}{cross_con_p}
    \row{Crossover}{Пересечение}{crossover}
    \row{Crossover adapter}{Перекрестный адаптер}{crossover_a}
    \row{Crossover cable}{Перекрестный кабель}{crossover_c}
    \rowit{Cut down}{синноним - punch down}{Врезка}{cut_down}


    \row{Delay}{Задержка}{delay}
    \row{Demo rack}{Демонстрационная стойка}{demo}
    \row{Direct buried}{Траншейный}{direct}
    \row{Dispersion}{Дисперсия}{dispersion}
    \row{Distribution frame, DF}{Кросс}{df}
    \row{Distributor}{Распределительный пункт}{distributor}
    \row{Dual fibre cable}{Двухволоконный оптический кабель}{dual}
    \rowNoLine{Duct}{Кабельный канал}{duct_1}
    \row{}{Воздухопровод}{duct_2}


    \row{Effective ground}{Эффективное заземление}{effective}
    \row{Electrical closet}{Электрощитовая}{electrical_closet}
    \row{Electromagnetic compatibility (EMC)}{Электромагнитная совместимость (ЭМС)}{emc}
    \row{Equal Level Far End Crosstalk (ELFEXT)}{Отношение затухания к однонаправленным наводкам}{elfext}
    \row{Entrance facility}{Ввод в здание}{entrance_f}
    \row{Entrance point}{Точка ввода}{entrance_p}
    \row{Entrance room or space for telecommunicaton}{Помещение ввода телекоммуникаций}{entrance_r}
    \row{Equipment cable}{Сетевой кабель}{equip_c}
    \row{Equipment room}{Комната для оборудования, аппаратная}{equip_r}
    \row{Ethernet}{{Ethernet, ""Эзернет""}}{ethernet}
    \row{Euromod}{Евромод}{euromod}
    \row{Exotermic weld}{Экзотермическая сварка}{exotermic}
    \row{Female connector}{Гнездовой разъем}{female}
    \row{Far End Crosstalk}{Однонаправленные наводки}{fext}
    \row{Fiber to the desk}{Абонентский волоконно-оптический канал}{fttd}
    \row{Fiber management system (FMS)}{Системы организации волокон (СОВ)}{fms}
    \row{Firestop}{Противопожарная заглушка}{firestop}
    \row{Floor distributor}{Распределительный пункт этажа}{floor}
    \row{Flush mount inserts}{Вставки заподлицо}{flush}
    \row{Foiled Twisted Pair (FTP)}{Фольгированная витая пара (ФВП)}{ftp}
    \row{Foiled and Braided Twisted Pair (FBTP)}{Фольгированная с оплеткой витая пара (ФОВП)}{fbtp}
    \row{Furniture cluster}{Блок рабочих мест}{furniture}
    \row{Generic cabling}{Структурированная кабельная система (СКС)}{generic_c}
    \row{Generic interface}{Структурированный интефейс}{generic_i}
    \row{Grade index fiber}{Градиентное волокно}{grade}
    \row{Grounding electrode}{Заземляющий электрод}{ground_e}
    \row{Grounding electrode conductor}{Проводник заземляющего электрода}{ground_e_c}
    \row{Horizontal cabe}{Горизонтальный кабель}{hori_cabe}
    \row{Horizontal cabling}{Горизонтальная подсистема}{hori_cabling}
    \rowit{Horizontal cross connect}{(ам. стандарт)}{Пункт коммутации горизонтальной подсистемы}{hori_x}
    \row{Hybrid cable}{Гибридный кабель}{hybrid}

\end{longtable}
