\setlength\parindent{1.25cm}

Учебное пособие "<Терминология структурированных кабельных сетей"> предназначено для
студентов, обучающихся по направлению "<Телекоммуникации"> по специальностям 20.09.00
"<Сети связи и системы коммутации">, 20.10.00 "<Многоканальные телекоммуникационные
системы">, 20.18.00 "<Защищенные системы связи">, 07.17.00 "<Физика и техника оптической
связи">, а также аспирантов по специальности 05.12.13 "<Системы, сети и устройства телекоммуникаций">.\par
Оно состоит из краткого англо-русского словаря терминов, снабженного их развернутыми 
толкованиями, и комплекса заданий, направленных на усвоение включенных в словарь терминов.\par
Вошедшие в словарь термины были отображены по принципу частотности употребления на основе анализа значительного
количества оригинальных англоязычных текстов по структурированным кабельным сетям.\par
Задания по ассимиляции терминов носят весьма разнообразный характер. В них термины
рассматриваются как в изолированном виде, так и в контексте и предназначены как для активного, так и для
пассивного усвоения. Тексты, предлагаемые в заданиях, взяты только из новейшей оригинальной англоязычной
литературы и не подвергались никакой авторской адаптации.\par
Учебное пособие может использоваться в аудиторной и внеаудиторной самостоятельной
работе студентов и аспирантов, указанных выше специальностей в качестве приложения к
учебнику Кожевниковой Т.В. "<Английский язык для университетов и институтов связи">-М;
Радио и связь, 2003. Кроме того, оно может быть полезно широкому кругу специалистов в 
области структурированных кабельных сетей, совершенствующих свой уровень владения
профессионально ориентированным английским языком.
